%%This is a very basic article template.
%%There is just one section and two subsections.
\documentclass{article}

\title{Programming Exercise}
\author{Sebastian Neumaier, 0925308 \\* Stefan Belk, 0925308}

\date{\today}

\begin{document}

\maketitle

\section{Task}

Formulate the k-MST problem as (mixed) integer linear programs (MILPs), based on
\begin{itemize}
\item Miller-Tucker-Zemlin subtour elimination constraints (\ref{mtz})
\item single commodity flows (\ref{scf})
\item multi commodity flows (\ref{mcf})
\end{itemize}
\section{Solution}

\subsection{MTZ: Miller-Tucker-Zemlin \label{mtz}}
In the Miller-Tucker-Zemlin formulation a variable $u_i$ is introduced to rank the selected nodes. The objective function is to minimize the costs $c_{ij}$ for each selected edge $x_{ij}$.
\setcounter{equation}{0}
\begin{equation}
min \sum_{(i,j) \in E} c_{ij} x_{ij}
\end{equation}
The rank of a node ranges between 0 and k. Therefore we need a constraint on the variable $u_i$.
\begin{equation}
\forall i \in N : 0 \le u_i \le k
\end{equation}
To prevent cycles we need a constraint on $u_i$ and $x_{ij}$. It ensures that no edge can be selected that is incident to two already selected nodes.
\begin{equation}
\forall (i,j) \in E : u_i + x_{ij} \le u_j + k * ( 1 - x_{ij})
\end{equation}
For simpler inplementation in our Java code we reformulated the equation above to:
\begin{equation}
\forall (i,j) \in E : u_i + x_{ij} - u_j \le k * ( 1 - x_{ij})
\end{equation}
The next contraint ensures that the root node is not part of the solution.
\begin{equation}
u_0 = 0
\end{equation}
As the weight (or costs) of each edge from the artificial root node 0 has value 0, we have to make sure that only one edge from the root node is selected.
\begin{equation}
\sum_{(0,j) \in E} x_{0j} = 0
\end{equation}

\subsection{SCF: Single Commodity Flows \label{scf}}
The objective function stays the same like in the Miller-Tucker-Zemlin formulation.
\setcounter{equation}{0}
\begin{equation}
min \sum_{(i,j) \in E} c_{ij} x_{ij}
\end{equation}
Flow has to be between 0 and k.
\begin{equation}
\forall (i,j) \in E : 0 \le f_{ij} \le k 
\end{equation}
Flow from node 0 to j is k.
\begin{equation}
\sum_{(0,j) \in E} f_{0j} = k
\end{equation}
Flow from node j to 0 is 0.
\begin{equation}
\sum_{(j,0) \in E} f_{j0} = 0
\end{equation}
Number of selected edges equals $k-1$.
\begin{equation}
\sum_{(i,j) \in E} x_{ij} = k - 1
\end{equation}
If a node has been selected the outgoing flow has to be equal to the sum of the incoming flows minus 1. 
\begin{equation}
\sum_{(i,x) \in E} f_{ix} - \sum_{(x,j) \in E} f_{xj} = min(1, \sum_{(i,x) \in E} f_{ix})
\end{equation}
Flow has to be 0 on edges that are not selected.
\begin{equation}
\forall (i,j) \in E : f_{ij} \le k*x_{ij}
\end{equation}
Only one outgoing edge from node 0 can be selected and no backwards edge.
\begin{equation}
\sum_{(0,j) \in E} x_{0j} = 1
\end{equation}
\begin{equation}
\sum_{(j,0) \in E} x_{j0} = 0
\end{equation}


\subsection{MCF: Multi Commodity Flows \label{mcf}}
In the MCF formulation we have a commodity for each node in the solution of the k-MST. Again the objective function is the same as in the previous formulations.
\setcounter{equation}{0}
\begin{equation}
min \sum_{(i,j) \in E} c_{ij} x_{ij}
\end{equation}
We allow only one outgoing edge from node 0.
\begin{equation}
\sum_{(i,j) \in E} x_{ij} = 1
\end{equation}
No edge from node j to 0 is selected.
\begin{equation}
\sum_{(j,0) \in E} x_{j0} = 0
\end{equation}
Number of selected edges equals $k-1$.
\begin{equation}
\sum_{(i,j) \in E} x_{ij} = k - 1
\end{equation}
We restrict the flow of commodity k from i to j to be a value between 0 and 1.
\begin{equation}
\forall (i,j) \in E : 0 \le f_{ij}^{l} \le 1
\end{equation}
If there is some flow from node i to j the edge (i,j) has to be selected.
\begin{equation}
\forall (i,j) \in E : f_{ij}^{l} \le x_{ij}
\end{equation}
The amount of commodities sent from node 0 to j has to be equal to k.
\begin{equation}
\sum_{(0,j) \in E} f_{0j}^{l} = k
\end{equation}
No flow is allowed to go back from node i to 0 in any case.
\begin{equation}
\sum_{(i,0) \in E} f_{i0}^{l} = 0
\end{equation}
Node 0 can send at most 1 commodity to a node.
\begin{equation}
\forall (0,j) \in E : f_{0j}^{l} \le 1
\end{equation}

\end{document}
