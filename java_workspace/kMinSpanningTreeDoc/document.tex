%%This is a very basic article template.
%%There is just one section and two subsections.
\documentclass{article}

\title{Programming Exercise}
\author{Sebastian Neumaier, 0925308 \\* Stefan Belk, 0750926}

\date{\today}

\begin{document}

\maketitle

\section{Task}

Formulate the k-MST problem as (mixed) integer linear programs (MILPs), based on
\begin{itemize}
\item Miller-Tucker-Zemlin subtour elimination constraints (\ref{mtz})
\item Single Commodity Flows (\ref{scf})
\item Multi Commodity Flows (\ref{mcf})
\end{itemize}
An undirected graph G = (V,E,w) with a non-negative weight function w was given. We have to find a minimum weight spanning tree with exactly k nodes. The following formulations have been solved with the ILOG CPLEX solver.
\section{Solution}

\subsection{MTZ: Miller-Tucker-Zemlin \label{mtz}}
In the Miller-Tucker-Zemlin formulation a variable $u_i$ is introduced to rank the selected nodes. The objective function is to minimize the costs $c_{ij}$ for each selected edge $x_{ij}$.
\setcounter{equation}{0}
\begin{equation}
min \sum_{(i,j) \in E} c_{ij} x_{ij}
\end{equation}
The rank of a node ranges between 0 and k. Therefore we need a constraint on the variable $u_i$.
\begin{equation}
\forall i \in N : 0 \le u_i \le k
\end{equation}
The node 0 has order 0. This ensures that the root node is not part of the solution.
\begin{equation}
u_0 = 0
\end{equation}
As the weight (or costs) of each edge from the artificial root node 0 has value 0, we allow only one outgoing edge from node 0.
\begin{equation}
\sum_{(0,j) \in E} x_{0j} = 1
\end{equation}
Number of selected edges equals $k-1$.
\begin{equation}
\sum_{(i,j) \in E} x_{ij} = k - 1
\end{equation}
The sum of all the orders has to be equal to k*(k+1)/2.
\begin{equation}
\sum_{n \in N} u_n = k*(k+1)/2
\end{equation}
And we need the sum of the selected nodes to be equal to k.
\begin{equation}
\sum_{n \in N} s_n = k
\end{equation}
A node is only allowed to have an order if it is selected.
\begin{equation}
\forall n \in N: u_n \le s_n
\end{equation}
Each node should have at most one incoming edge.
\begin{equation}
\forall n \in N: \sum_{(i,n) \in E} x_{in} = 1
\end{equation}
If an edge is selected, then both nodes have to have an order.
\begin{equation}
\forall (i,j) \in E, j \not= 0: x_{ij} \le u_i
\end{equation}
To prevent cycles we need a constraint on $u_i$ and $x_{ij}$. It ensures that no edge can be selected that is incident to two already selected nodes.
\begin{equation}
\forall (i,j) \in E : u_i + x_{ij} \le u_j + k * ( 1 - x_{ij})
\end{equation}
For simpler inplementation in our Java code we reformulated the equation above to:
\setcounter{equation}{10}
\begin{equation}
\forall (i,j) \in E : u_i + x_{ij} - u_j \le k * ( 1 - x_{ij})
\end{equation}
If an edge has been selected then its back-edge cannot be selected.
\begin{equation}
\forall (i,j) \in E: x_{ij} + x_{ji} \le 1
\end{equation}

\subsection{SCF: Single Commodity Flows \label{scf}}
The objective function stays the same like in the Miller-Tucker-Zemlin formulation.
\setcounter{equation}{0}
\begin{equation}
min \sum_{(i,j) \in E} c_{ij} x_{ij}
\end{equation}
Flow has to be between 0 and k.
\begin{equation}
\forall (i,j) \in E : 0 \le f_{ij} \le k 
\end{equation}
Flow from node 0 to j is k.
\begin{equation}
\sum_{(0,j) \in E} f_{0j} = k
\end{equation}
Flow from node j to 0 is 0.
\begin{equation}
\sum_{(j,0) \in E} f_{j0} = 0
\end{equation}
Number of selected edges equals $k-1$.
\begin{equation}
\sum_{(i,j) \in E} x_{ij} = k - 1
\end{equation}
If a node has been selected the outgoing flow has to be equal to the sum of the incoming flows minus 1. 
\begin{equation}
\sum_{(i,x) \in E} f_{ix} - \sum_{(x,j) \in E} f_{xj} = min(1, \sum_{(i,x) \in E} f_{ix})
\end{equation}
Flow has to be 0 on edges that are not selected.
\begin{equation}
\forall (i,j) \in E : f_{ij} \le k*x_{ij}
\end{equation}
Only one outgoing edge from node 0 can be selected and no backwards edge.
\begin{equation}
\sum_{(0,j) \in E} x_{0j} = 1
\end{equation}
\begin{equation}
\sum_{(j,0) \in E} x_{j0} = 0
\end{equation}
If an edge has been selected then its back-edge cannot be selected.
\begin{equation}
\forall (i,j) \in E: x_{ij} + x_{ji} \le 1
\end{equation}

\subsection{MCF: Multi Commodity Flows \label{mcf}}
In the MCF formulation we have a commodity for each node in the solution of the k-MST. Again the objective function is the same as in the previous formulations.
\setcounter{equation}{0}
\begin{equation}
min \sum_{(i,j) \in E} c_{ij} x_{ij}
\end{equation}
We allow only one outgoing edge from node 0.
\begin{equation}
\sum_{(i,j) \in E} x_{ij} = 1
\end{equation}
No edge from node j to 0 is selected.
\begin{equation}
\sum_{(j,0) \in E} x_{j0} = 0
\end{equation}
Number of selected edges equals $k-1$.
\begin{equation}
\sum_{(i,j) \in E} x_{ij} = k - 1
\end{equation}
We restrict the flow of commodity k from i to j to be a real value between 0 and 1.
\begin{equation}
\forall (i,j) \in E, \forall n \in N : 0 \le f_{ij}^{n} \le 1
\end{equation}
If there is some flow from node i to j the edge (i,j) has to be selected.
\begin{equation}
\forall (i,j) \in E, \forall n \in N : f_{ij}^{n} \le x_{ij} 
\end{equation}
The amount of commodities sent from node 0 to j has to be equal to k.
\begin{equation}
\forall n \in N: \sum_{(0,j) \in E} f_{0j}^{n} = k
\end{equation}
No flow is allowed to go back from node i to 0 in any case.
\begin{equation}
\forall n \in N: \sum_{(i,0) \in E} f_{i0}^{n} = 0
\end{equation}
Node 0 can send at most 1 commodity to a node.
\begin{equation}
\forall (0,j) \in E : f_{0j}^{n} = k
\end{equation}
All nodes together receive k commodities.
\begin{equation}
\sum_{n \in N} \sum_{(i,j) \in E} f_{ij}^{n} \le 1
\end{equation}
A single node can receive at most 1 commodity.
\begin{equation}
\forall n \in N \backslash \{0\} : \sum_{i,(i,j) \in E} f_{ij}^{n} \le 1
\end{equation}
The sum of incoming flow has to equal to the sum of outgoing flow.
\begin{equation}
\forall j,n \in N \backslash \{0\}, j \not= n : \sum_{i,(i,j) \in E} f_{ij}^{n} = \sum_{i,(j,i) \in E} f_{ji}^{n}
\end{equation}


\section{Results}

\subsection{Runtimes (in seconds)}
\begin{tabular}{| l l l l l |}
\hline
graph	&	k	&	MTZ	&	SCF  	&	MCF	\\ \hline \hline
g01	&	2	&	0.34	&	0.36	&	0.28	\\
	&	5	&	0.36 	&	0.37	&	0.31	\\ \hline
g02	&	4	&	0.48	&	0.48	&	0.81	\\
	&	10	&	0.44	&	1.39	&	0.69	\\ \hline
g03	&	10	&	0.94	&	3.80	&	3.91	\\
	&	25	&	3.22	&	6.23	&	9.81	\\ \hline
g04	&	14	&	1.42	&	39.95	&	165.36	\\
	&	35	&	3.78	&	77.09	&	138.91	\\ \hline
g05	&	20	&	7.09	&	82.00	&	317.36	\\
	&	50	&	18.05	&	425.69	&	186.80	\\ \hline
g06	&	40	&	408.39	&			&		\\
	&	100	&	319.78  &			&		\\ \hline
g07	&	60	&	96.22	&			&		\\
	&	150	&			&			&		\\ \hline
g08	&	80	&			&			&		\\
	&	200	&			&			&		\\ \hline
\end{tabular}

\subsection{MTZ: Miller-Tucker-Zemlin}
\begin{tabular}{| l l l l l |}
\hline
graph	&	k	&	time	&	b-a-b nodes &	value	\\ \hline \hline 
g01	&	2	&	0.343750016	&	0			&	46	\\
	&	5	&	0.359375008	&	0			&	477	\\ \hline
g02	&	4	&	0.484375008	&	35			&	373	\\
	&	10	&	0.4375		&	0			&	1390	\\ \hline
g03	&	10	&	0.937500032	&	198			&	725	\\
	&	25	&	3.218749952	&	1618		&	3074	\\ \hline
g04	&	14	&	1.421874944	&	39			&	909	\\
	&	35	&	3.781250048	&	1286		&	3292	\\ \hline
g05	&	20	&	7.09374976	&	1349		&	1235	\\
	&	50	&	18.04687564	&	5108		&	4898	\\ \hline
g06	&	40	&	408.3906314	&	76001		&	2068	\\
	&	100	&	319.78124083&	88563		&	6705	\\ \hline
g07	&	60	&	96.218750976&	4037	&	1335	\\
	&	150	&				&		&	4534	\\ \hline
g08	&	80	&				&		&	1620	\\
	&	200	&				&		&	5787	\\ \hline
\end{tabular}

\subsection{SCF: Single Commodity Flows}
\begin{tabular}{| l l l l l |}
\hline
graph	&	k	&	time		&	b-a-b nodes &	value	\\ \hline \hline
g01	&	2	&	0.359375008	&	0	&	46	\\
	&	5	&	0.375		&	60	&	477	\\ \hline
g02	&	4	&	0.484375008	&	201	&	373	\\
	&	10	&	1.390625024	&	1654	&	1390	\\ \hline
g03	&	10	&	3.796875008	&	985	&	725	\\
	&	25	&	6.234375168	&	1582	&	3074	\\ \hline
g04	&	14	&	39.953125376	&	11296	&	909	\\
	&	35	&	77.093748736	&	24180	&	3292	\\ \hline
g05	&	20	&	82.000003072	&	17082	&	1235	\\
	&	50	&	425.687515336		&	66792	&	4898	\\ \hline
g06	&	40	&			&		&	2068	\\
	&	100	&			&		&	6705	\\ \hline
g07	&	60	&			&		&	1335	\\
	&	150	&			&		&	4534	\\ \hline
g08	&	80	&			&		&	1620	\\
	&	200	&			&		&	5787	\\ \hline
\end{tabular}


\subsection{MCF: Multi Commodity Flows}
\begin{tabular}{| l l l l l |}
\hline
graph	&	k	&	time in s	&	b-a-b nodes &	value	\\ \hline \hline
g01	&	2	&	0.28125 	&	0	&	46	\\
	&	5	&	0.3125 		&	5	&	477	\\ \hline
g02	&	4	&	0.8125 		&	22	&	373	\\
	&	10	&	0.6875 		&	0	&	1390	\\ \hline
g03	&	10	&	3.90625 	&	0	&	725	\\
	&	25	&	9.8125 		&	0	&	3074	\\ \hline
g04	&	14	&	165.359375 	&	195	&	909	\\
	&	35	&	138.90625 	&	13	&	3292	\\ \hline
g05	&	20	&	317.359375 	&	27	&	1235	\\
	&	50	&	186.796875 	&	0	&	4898	\\ \hline
g06	&	40	&			&		&	2068	\\
	&	100	&			&		&	6705	\\ \hline
g07	&	60	&			&		&	1335	\\
	&	150	&			&		&	4534	\\ \hline
g08	&	80	&			&		&	1620	\\
	&	200	&			&		&	5787	\\ \hline
\end{tabular}




\section{Conclusion}
In our implementation MTZ proved to be the best method. It performed far better then SCF and MCF. Aditionally SCF performed better then MCF in most instances.

\end{document}
