\section{Single commodity flows (SCF)}

\subsection{problem description}
SCF (Single commodity flow) uses outgoing flow from an artificial node. Those nodes receiving any flow larger than zero are part of the k-MST solution

\subsection{our approach}
We are introducing a decision variable $x_{ij} \forall x_{ij} \in \{0,1\}$ which determines if the edge $(i,j)$ is part of the k-MST solution.

\begin{equation}
  min \sum_{(i,j) \in E, \ i,j \not =  0} w_{ij} * x_{ij} 
\end{equation}

\subsection{variables and constraints}

The artificial node $0$ sends out exactly $k$ commodities.
\begin{equation}
  \sum_{(0,j) \in E} f_{0j} = k 
\end{equation}

A k-MST with $k$ nodes needs $k-1$ edges.
\begin{equation}
  \sum_{(i,j) \in E, \ i,j \not =  0} x_{ij} = k-1 
\end{equation}

The variable $f_{ij}$ holds the flow on the edge $(i,j)$. 
The following equation ensures that each node within the k-MST
receives one commodity and forwards the rest of the commodities. 
All other nodes do not receive a commodity.
\begin{equation}
  \sum_{u, (v,u) \in E} f_{vu} - \sum_{u, (u,v) \in E} f_{uv} = \min (1, \sum_{u, (v,u) \in E} f_{vu}) \qquad  \forall v \not = 0, v \ \in N 
\end{equation}

If we have a flow on the edge $(i,j)$, then this edge has to be
in the solution of the k-MST.
\begin{equation}
  f_{ij} \leq k * x_{ij} \qquad \forall (i,j) \in E 
\end{equation}

We ensure that only one edge from the artificial node to any
other node is selected.
\begin{equation}
  \sum_{(0,j) \in E} x_{0j} = 1 
\end{equation}

\begin{table} 
\small
\centering
\begin{tabular}{ccccc}
\hline
test     & k & objective      & running & branch-and-bound \\
instance &   & function value & time    & nodes \\
\hline
data/g01.dat 	& 1 & 2 & 0.3 & 4 \\

\hline
\end{tabular}
\caption{explain captions}
\label{tbl:scf_slow}
\end{table}

\subsection{a faster solution}

After strengthening our constraints we were able to obtain the solution much faster than before.\\
%ende von Nick

If the edge $(i,j)$ is in the solution, then also both end nodes $i$ and $j$ 
have to be in the solution.
\begin{equation}
  x_{ij} \leq y_{i} \qquad \forall (i,j) \in E
  \label{lbl:xy1}
\end{equation}

\begin{equation}
  x_{ij} \leq y_{j} \qquad \forall (i,j) \in E
  \label{lbl:xy2}
\end{equation}

\begin{equation}
 y_{i} + x_{ij} + x_{ji} \leq y_{j} + 1 \qquad \forall (i,j) \in E
 \label{lbl:xy3}
\end{equation}

Exactly k+1 nodes (with artificial node) must be selected.
\begin{equation}
  \sum_{i \in N} y_i = k + 1
  \label{lbl:yk}
\end{equation}

\begin{equation}
  0 \leq f_{ij} \leq k \qquad \forall (i,j) \in E
\end{equation}

\begin{equation}
  x_{ij} \in \{0,1\} \qquad \forall (i,j) \in E
\end{equation}

\begin{equation}
  y_i \in \{0,1\} \qquad \forall i \in N
\end{equation}


\begin{table} 
\small
\centering
\begin{tabular}{ccccc}
\hline
test     & k & objective      & running & branch-and-bound \\
instance &   & function value & time    & nodes \\
\hline
data/g01.dat		&2	&46	&0.02	&0	\\ 
data/g01.dat		&5	&477	&0.02	&0	\\ 
data/g01.dat		&10	&2692	&0.04	&0	\\ 
data/g02.dat		&4	&373	&0.04	&0	\\ 
data/g02.dat		&10	&1390	&0.34	&49	\\ 
data/g02.dat		&20	&5565	&0.28	&53	\\ 
data/g03.dat		&10	&725	&0.29	&0	\\ 
data/g03.dat		&25	&3074	&0.72	&0	\\ 
data/g03.dat		&50	&11400	&0.72	&34	\\ 
data/g04.dat		&14	&909	&0.64	&63	\\ 
data/g04.dat		&35	&3292	&6.11	&1228	\\ 
data/g05.dat		&20	&1235	&10.78	&2474	\\ 
data/g05.dat		&50	&4898	&12.62	&1189	\\ 
data/g06.dat		&40	&2068	&1025.64	&28212	\\ 
data/g06.dat		&100	&6705	&155.06	&4608	\\

\hline
\end{tabular}
\caption{explain captions}
\label{tbl:scf_fast}
\end{table}
