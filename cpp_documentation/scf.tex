\section{Single commodity flows (SCF)}

\subsection{Our Approach}
The Single commodity flow uses an outgoing flow from an artificial node. This flow is set to the amount of nodes we want our spanning tree to have. Each time the flow passes a node the sum of the outgoing flows will be reduced by one. Any node receiving any flow larger than zero and every edge having a flow larger then zero are part of the spanning tree.
\\
\\
For this we are introducing a decision variable $x_{ij} \forall x_{ij} \in \{0,1\}$ which determines if the edge $(i,j)$ is part of the k-MST solution.
\\
\\
Our objective function is the spanning tree with the lowest costs.

\begin{equation}
  min \sum_{(i,j) \in E, \ i,j \not =  0} w_{ij} * x_{ij} 
\end{equation}

\subsection{variables and constraints}

The variable $x$ can have the value 0 if the edge is not and 1 if the edge is in the solution. 

\begin{equation}
  x_{ij} \in \{0,1\} \qquad \forall (i,j) \in E
\end{equation}
\\
For more strict constraints we need another variable $y$ which has the value 0 if a node $i$ is not and 1 if it is in the solution.

\begin{equation}
  y_i \in \{0,1\} \qquad \forall i \in V
\end{equation}
\\
For the SCF we need a flow variable $f_{ij}$ which can have a value between 0 and k where $f_{ij}$ holds the flow on the edge $(i,j)$. 

\begin{equation}
  0 \leq f_{ij} \leq k \qquad \forall (i,j) \in E
\end{equation}
\\
The artificial node $0$ sends out a flow of value $k$.
\begin{equation}
  \sum_{(0,j) \in E} f_{0j} = k 
\end{equation}
\\
The artificial node $0$ is not allowed to receive any flow.
\begin{equation}
  \sum_{(i,0) \in E} f_{i0} = 0 
\end{equation}
\\
We ensure that only one edge from the artificial node to any.
other node is selected.
\begin{equation}
  \sum_{(0,j) \in E} x_{0j} = 1 
\end{equation}
\\
A minimum spanning tree with $k$ nodes must have $k-1$ edges.
\begin{equation}
  \sum_{(i,j) \in E, \ i,j \not =  0} x_{ij} = k-1 
\end{equation}
\\
The following equation states that the sum of all incoming flows minus the sum of all outgoing flows must be 0 or 1. This ensures that each time a node is selected, a commodity is consumed or no flow is incoming or outgoing.
\begin{equation}
  \sum_{u, (v,u) \in E} f_{vu} - \sum_{u, (u,v) \in E} f_{uv} = \min (1, \sum_{u, (v,u) \in E} f_{vu}) \qquad  \forall v \not = 0, v \ \in V
\end{equation}
\\
If we have a flow on the edge $(i,j)$, then this edge has to be selected and therefore must be in the solution

\begin{equation}
  f_{ij} \leq k * x_{ij} \qquad \forall (i,j) \in E 
\end{equation}
\\
If the edge $(i,j)$ is in the solution, then also both end nodes $i$ and $j$ 
have to be in the solution.
\begin{equation}
  x_{ij} \leq y_{i} \qquad \forall (i,j) \in E
  \label{lbl:xy1}
\end{equation}

\begin{equation}
  x_{ij} \leq y_{j} \qquad \forall (i,j) \in E
  \label{lbl:xy2}
\end{equation}
\\
The following equation ensures that only one direction of an edge is in the solution.

\begin{equation}
 y_{i} + x_{ij} + x_{ji} \leq y_{j} + 1 \qquad \forall (i,j) \in E
 \label{lbl:xy3}
\end{equation}
\\
Exactly k+1 nodes, including the artificial node, must be selected.
\begin{equation}
  \sum_{i \in V} y_i = k + 1
  \label{lbl:yk}
\end{equation}
\\
To strenghten the constraints and to further ensure that no clycles will be in the solution we need to go sure that each node has exactly $1$ or none incoming edges. This equation is not necessary but will increase the performance dramatically.

\begin{equation}
  \sum_{i \in V} {x_{in}} \leq 1 \qquad \forall n \in V \backslash (0)
\end{equation}


\begin{table} 
\small
\centering
\begin{tabular}{ccccc}
\hline
test     & k & objective      & running & branch-and-bound \\
instance &   & function value & time    & nodes \\
\hline
data/g01.dat		& 2	& 46	& 0	& 0	\\ 
data/g01.dat		& 5	& 477	& 0.01	& 0	\\ 
data/g02.dat		& 4	& 373	& 0.02	& 0	\\ 
data/g02.dat		& 10	& 1390	& 0.03	& 0	\\ 
data/g03.dat		& 10	& 725	& 0.06	& 0	\\ 
data/g03.dat		& 25	& 3074	& 0.07	& 0	\\ 
data/g04.dat		& 14	& 909	& 0.15	& 17	\\ 
data/g04.dat		& 35	& 3292	& 2.18	& 1476	\\ 
data/g05.dat		& 20	& 1235	& 0.36	& 20	\\ 
data/g05.dat		& 50	& 4898	& 0.4	& 7	\\ 
data/g06.dat		& 40	& 2068	& 47.71	& 2225	\\ 
data/g06.dat		& 100	& 6705	& 30.25	& 2964	\\ 

\hline
\end{tabular}
\caption{explain captions}
\label{tbl:scf_fast}
\end{table}
