\section{Multi commodity flow (MCF)}
\subsection{Our Approach}
The Multi Commodity Flow sends out $k$ different flows each directed to a specific node which will be in the k-MST solution. Each flow can pass arbitrarily many other flows on it's path.
\\
\\
For this problem we introduce the same deciscion variable $x_{ij} \forall x_{ij} \in \{0,1\}$ and the same objective function as we did for the SCF.\\
\\
\begin{equation}
  min \sum_{(i,j) \in E, \ i,j \not =  0} w_{ij} * x_{ij} 
\end{equation}

\subsection{Variables and Constraints}

The variable $x$ can have the value 0 if the edge is not and 1 if the edge is in the solution. 

\begin{equation}
  x_{ij} \in \{0,1\} \qquad \forall (i,j) \in E
\end{equation}
\\
We need another variable $y$ which has the value 0 if a node $i$ is not and 1 if it is in the solution.

\begin{equation}
  y_i \in \{0,1\} \qquad \forall i \in V
\end{equation}
\\
For the MCF we need, like in the SCF, a flow variable, but we have $n$ different flows where each flow $f_{ij}^n$ has a value beween 0 and 1 and is directed to a specific node $n$.

\begin{equation}
  0 \leq f_{ij}^n \leq 1 \qquad \forall (i,j) \in E, \forall n \in V
\end{equation}
\\
The following equation ensures that the artificial node $0$ sends out a flow to each different node $n$ with a value between 0 and 1.

\begin{equation}
  \sum_{j, j \neq 0}  f_{0j}^n \leq 1 \qquad \forall n \in V \backslash \{0\}
\end{equation}
\\
This equation guarantees that exactly k flows are sent to graph.

\begin{equation}
  \sum_{(0,j) \in E}  f_{0j}^n = k \qquad \forall n \in V
\end{equation}
\\
Here we assure that the sum of all flows which are incoming in the destiny node must have a value between 0 and 1.

\begin{equation}
  \sum_{i, i \neq n}  f_{in}^n \leq 1 \qquad \forall n \in V \backslash \{0\}
\end{equation}
\\
We need go sure, that the sum of all incoming flows on the destination nodes must be equal to the sum of the artificial node which has sent a value of $k$.

\begin{equation}
  \sum_{n=1}^V \sum_{i, (i,n) \in E}  f_{in}^n= k
\end{equation}
\\
Each flow which is incoming in an node other then the destination node must be sent further.

\begin{equation}
  \sum_{i, (i,j) \in E}  f_{ij}^n -  \sum_{i, (j,i) \in E}  f_{ji}^n = 0 \qquad \forall n \in V \backslash \{0\}, \forall j, j \neq n
\end{equation}
\\
We need to go sure, that if a flow greater then 0 is on an edge than the edge must be selected.

\begin{equation}
  f_{ij}^n \leq x_{ij} \qquad \forall (i,j) \in E, \forall n \in V
\end{equation}
\\
Further we need to ensure that the artificial node $0$ sends all k different flows over one specific node. Therefore only one edge can be selected.

\begin{equation}
  \sum_{(0, j) \in E}  x_{0j} = 1
\end{equation}
\\
To be sure that each solution is truly a minimum spanning tree we need to be sure that there are no cylces. We accomplish that by the following equation that guarantees us that there can only be $k-1$ edges selected.

\begin{equation}
  \sum_{(i, j) \in E, i,j \neq 0}  x_{ij} = k-1
\end{equation}
\\
If the edge $(i,j)$ is in the solution, then also both end nodes $i$ and $j$ 
have to be in the solution.
\begin{equation}
  x_{ij} \leq y_{i} \qquad \forall (i,j) \in E
  \label{lbl:xy1}
\end{equation}

\begin{equation}
  x_{ij} \leq y_{j} \qquad \forall (i,j) \in E
  \label{lbl:xy2}
\end{equation}
\\
The following equation ensures that only one direction of an edge is in the solution.

\begin{equation}
 y_{i} + x_{ij} + x_{ji} \leq y_{j} + 1 \qquad \forall (i,j) \in E
 \label{lbl:xy3}
\end{equation}
\\
Exactly k+1 nodes, including the artificial node, must be selected.
\begin{equation}
  \sum_{i \in V} y_i = k + 1
  \label{lbl:yk}
\end{equation}
\\
To strenghten the constraints and to further ensure that no clycles will be in the solution we need to go sure that each node has exactly $1$ or none incoming edges. This equation is not necessary but will increase the performance.

\begin{equation}
  \sum_{i \in V} {x_{in}} \leq 1 \qquad \forall n \in V \backslash (0)
\end{equation}


\begin{table} 
\small
\centering
\begin{tabular}{ccccc}
\hline
test     & k & objective      & running & branch-and-bound \\
instance &   & function value & time    & nodes \\
\hline
data/g01.dat		& 2	& 46	& 0.03	& 0	\\ 
data/g01.dat		& 5	& 477	& 0.02	& 0	\\ 
data/g02.dat		& 4	& 373	& 0.33	& 11	\\ 
data/g02.dat		& 10	& 1390	& 0.13	& 0	\\ 
data/g03.dat		& 10	& 725	& 3.31	& 0	\\ 
data/g03.dat		& 25	& 3074	& 8.18	& 0	\\ 
data/g04.dat		& 14	& 909	& 91.96	& 39	\\ 
data/g04.dat		& 35	& 3292	& 76.38	& 0	\\ 
data/g05.dat		& 20	& 1235	& 368.58	& 0	\\ 
data/g05.dat		& 50	& 4898	& 362.9	& 0	\\ 

\hline
\end{tabular}
\caption{explain captions}
\label{tbl:scf_fast}
\end{table}
