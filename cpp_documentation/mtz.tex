\section{Miller Tucker Zemlin (MTZ)}

\subsection{Our Approach}
Miller Tucker Zemlin uses sequential node ordering. We use again an artificial node $0$ starting with the order $0$ and increment the order every time we reach a new node by one.\\
\\ 
Again we introduce the same deciscion variable $x_{ij} \forall x_{ij} \in \{0,1\}$ and the same objective function as we did for the SCF.\\

\begin{equation}
  min \sum_{(i,j) \in E , \ i,j \not = 0} {w_{ij} * x_{ij}}
\end{equation}


\subsection{Variables and Constraints}
The variable $x$ can have the value 0 if the edge is not and 1 if the edge is in the solution. 

\begin{equation}
  x_{ij} \in \{0,1\} \qquad \forall (i,j) \in E
\end{equation}
\\
We need another variable $y$ which has the value 0 if a node $i$ is not and 1 if it is in the solution.

\begin{equation}
  y_i \in \{0,1\} \qquad \forall i \in V
\end{equation}
\\
For MTZ we introduce a new variable $u_i$ which is the order of the every node $i$. It can have a value between 0 and $k$, because the highest order there can be is exactly $k$.

\begin{equation}
  0 \leq u_i \leq k \qquad \forall i \in V \backslash (0) 
\end{equation}
\\
The artifical node $0$ has the order $0$.

\begin{equation}
  u_0 = 0 
\end{equation}
\\
The artificial node can have only one outgoing edge selected.

\begin{equation}
  \sum_{(0,j) \in E} x_{0j} = 1 
\end{equation}
\\
This equation ensures that an edge can only exist between a node with a lower order than the targeting node. This guarantees us that there will be no cycles.

\begin{equation}
  u_{i} + x_{ij} \leq u_{j} + k*(1- x_{ij}) \qquad \forall (i,j) \in E 
\end{equation}
\\
To ensure that no order will be skipped we restict the sum of orders.

\begin{equation}
  \sum_{i \in V} u_i = \frac{k*(k+1)}{2} 
\end{equation}
\\
We also need to ensure that if a node has a order greater than $0$ then the node has to be selected.

\begin{equation}
  u_i \leq k*y_i \qquad \forall i \in V
\end{equation}
\\
The k-MST solution must have exactly $k-1$ edges selected.

\begin{equation}
  \sum_{(i,j) \in E, \ i,j \not =  0} x_{ij} = k-1 
\end{equation}
\\
Further the solution must have exactly $k$ nodes selected.

\begin{equation}
  \sum_{i \in V, i \neq 0} y_i = k
\end{equation}
\\
To strenghten the constraints and to further ensure that no clycles will be in the solution we need to go sure that each node has exactly $1$ or none incoming edges.

\begin{equation}
  \sum_{i \in V} {x_{in}} \leq 1 \qquad \forall n \in V \backslash (0)
\end{equation}
\\
A node can only have an outgoing edge selected, if the node has a order greater than $0$.

\begin{equation}
  x_{ij} \leq u_i \qquad \forall (i,j) \in E, i \neq 0
\end{equation}
\\
A node can only have an incoming edge selected, if the node has a order greater than $0$.

\begin{equation}
  x_{ij} \leq u_j \qquad \forall (i,j) \in E, j \neq 0
\end{equation}
\\
The following equation ensures that only one direction of an edge is in the solution.

\begin{equation}
 y_{i} + x_{ij} + x_{ji} \leq y_{j} + 1 \qquad \forall (i,j) \in E, i \neq 0, j \neq 0
\end{equation}
\\

\begin{table} 
\small
\centering
\begin{tabular}{ccccc}
\hline
test     & k & objective      & running & branch-and-bound \\
instance &   & function value & time    & nodes \\
\hline
data/g01.dat		& 2	& 46	& 0.03	& 0	\\ 
data/g01.dat		& 5	& 477	& 0.17	& 0	\\ 
data/g01.dat		& 10	& 2692	& 0.41	& 0	\\ 
data/g02.dat		& 4	& 373	& 0.72	& 0	\\ 
data/g02.dat		& 10	& 1390	& 1.14	& 51	\\ 
data/g02.dat		& 20	& 5565	& 2.34	& 1197	\\ 
data/g03.dat		& 10	& 725	& 2.83	& 57	\\ 
data/g03.dat		& 25	& 3074	& 11.91	& 2306	\\ 
data/g03.dat		& 50	& 11400	& 235.76	& 136934	\\ 
data/g04.dat		& 14	& 909	& 8.62	& 151	\\ 
data/g04.dat		& 35	& 3292	& 18.83	& 2417	\\ 
data/g05.dat		& 20	& 1235	& 12.92	& 369	\\ 
data/g05.dat		& 50	& 4898	& 56.91	& 6930	\\ 
data/g06.dat		& 40	& 2068	& 171.72	& 13414	\\ 
data/g06.dat		& 100	& 6705	& 892.2	& 49561	\\ 

\hline
\end{tabular}
\caption{explain captions}
\label{tbl:scf_fast}
\end{table}
