\section{Miller Tucker Zemlin (MTZ)}
\subsection{problem description}
MTZ (Miller Tucker Zemlin) uses sequential node ordering (instead of an artificial node). 

\subsection{our approach}
We are introducing a variable $u_i$ which determines the order of the nodes in the k-MST solution.\\

\begin{equation}
  min \sum_{(i,j) \in E , \ i,j \not = 0} {w_{ij} * x_{ij}}
\end{equation}


\subsection{variables and constraints}

As we already explained above the following constraint with the variable
$u_i$ determines the order between the nodes and ensures that we do
not get a cycle.
\begin{equation}
  u_{i} + x_{ij} \leq u_{j} + k*(1- x_{ij}) \qquad \forall (i,j) \in E 
\end{equation}
Since we want only $k$ nodes in the solution, all $u_i$'s have to be
at most $k$.
\begin{equation}
  0 \leq u_i \leq k \qquad \forall i \in N \backslash (0) 
\end{equation}
The artifical node has the order $0$ and it has exactly one outgoing edge.
\begin{equation}
  u_0 = 0 
\end{equation}
\begin{equation}
  \sum_{(0,j) \in E} x_{0j} = 1 
\end{equation}
The solution must have exactly $k-1$ edges.
\begin{equation}
  \sum_{(i,j) \in E, \ i,j \not =  0} x_{ij} = k-1 
\end{equation}
To further tighten the constraints, we restrict the sum of the orders
from the nodes in the solution.
\begin{equation}
  \sum_{i \in N} u_i = \frac{k*(k+1)}{2} 
\end{equation}
Furthermore, we also know that we need exactly $k$ nodes with an
order $u_i > 0$ in the solution.
\begin{equation}
  u_i \leq k*y_i \qquad \forall i \in N
\end{equation}
\begin{equation}
  \sum_{i \in N} y_i = k
\end{equation}
Since we have arcs in our formulation, we know that each node 
has at most one incoming arc in the solution. 
\begin{equation}
  \sum_{i \in N} {x_{ij}} \leq 1 \qquad \forall j \in N \backslash (0)
\end{equation}
Whenever any arc is chosen, the connected nodes must have an order.
\begin{equation}
  x_{ij} \leq u_i \qquad \forall (i,j) \in E, i \neq 0
\end{equation}
\begin{equation}
  x_{ij} \leq u_j \qquad \forall (i,j) \in E, j \neq 0
\end{equation}
If the node $i$ and the edge $(i,j)$ is in the solution, then also the node $j$ has to be in the solution.
\begin{equation}
 y_{i} + x_{ij} + x_{ji} \leq y_{j} + 1 \qquad \forall (i,j) \in E, i \neq 0, j \neq 0
 \label{lbl:mtzxy3}
\end{equation}

\begin{equation}
x_{ij} \in \{0,1\} \qquad \forall (i,j) \in E
\end{equation}
\begin{equation}
  y_i \in \{0,1\} \qquad \forall i \in N
\end{equation}
